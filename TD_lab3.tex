\documentclass[a4paper,12pt,notitlepage]{article}

\usepackage[svgnames]{xcolor}
\usepackage{amsmath, amssymb, amsthm}
\usepackage{mathtools}
\usepackage[math]{fontspec}
\usepackage[nosingleletter, lastparline]{impnattypo}
\usepackage[polish]{babel}
\usepackage{bookmark}
\usepackage[margin=1in]{geometry}
\usepackage[babel=true, tracking=true]{microtype}
\usepackage{minted}
\usepackage{array}
\usepackage{colortbl}

\renewcommand*{\thesection}{\Alph{section}}

\definecolor{bg}{rgb}{0.95,0.95,0.95}
\setminted{frame=single, bgcolor=bg, breaklines=true}

\linespread{1.3}
\setlength{\parindent}{0pt}

\title{\textbf{TD -- laboratorium 3}}
\author{Piotr Rogulski 305867 \\ Szymon Sieradzki 305881}
\date{\today}

\IfFontExistsTF{JetBrainsMono-Regular}{
    \setmonofont{JetBrainsMono}[
        UprightFont = *-Light,
        BoldFont = *-Regular,
        ItalicFont = *-Light-Italic,
        Scale = MatchLowercase
    ]
}{}

\begin{document}

\maketitle
\section*{Schemat adresacji}

\begin{table}[!htb]
    \caption{Adresacja interfejsów}
    \centering
    \begin{tabular}{*4c}
        \hline\hline
            \textbf{Router} & \textbf{Interfejs} & \textbf{Adres IP} & \textbf{Podsieć} \\
        \hline
            \rowcolor{bg}
            R1 & e0/0 & 192.168.11.1  & 192.168.11.0/30 \\
        \hline
            \rowcolor{bg}
            R2 & e0/0 & 192.168.11.2  & 192.168.11.0/30 \\
               & e0/1 & 192.168.10.1  & 192.168.10.0/30 \\
               & e0/2 & 192.168.10.5  & 192.168.10.4/30 \\
        \hline
            R3 & e0/0 & 192.168.10.1  & 192.168.10.0/30 \\
               & e0/2 & 192.168.10.6  & 192.168.10.4/30 \\
               & e0/3 & 192.168.10.17 & 192.168.10.16/30 \\
        \hline
            R4 & e0/1 & 192.168.10.2  & 192.168.10.0/30 \\
               & e0/2 & 192.168.10.5  & 192.168.10.4/30 \\
               & e0/3 & 192.168.10.18 & 192.168.10.16/30 \\
        \hline
            R5 & e0/0 & 192.168.10.2  & 192.168.10.0/30 \\
               & e0/2 & 192.168.10.6  & 192.168.10.4/30 \\
        \hline\hline
    \end{tabular}
\end{table}

\begin{table}[!htb]
    \caption{Adresacja interfejsów loopback}
    \centering
    \begin{tabular}{*2c}
        \hline\hline
            \textbf{Router} & \textbf{Loopback IP} \\
        \hline
            R1 & 192.168.0.1 \\
            R2 & 192.168.0.2 \\
            R3 & 192.168.0.3 \\
            R4 & 192.168.0.4 \\
            R5 & 192.168.0.5 \\
        \hline\hline
    \end{tabular}
\end{table}

\section{Podstawowa konfiguracja urządzenia}

W początkowej konfiguracji urządzeń każdemu interfejsowi na każdym routerze przypisano adres IP komendą \mintinline{text}{ip address <ip address> <mask>} a następnie potwierdzono pomyślne przypisanie komendami \mintinline{text}{show cdp neighbors} oraz \mintinline{text}{show ip route}.

Na każdym z routerów wpisów w tablicy trasowania jest tyle, ile skonfigurowanych interfejsów, każdy w innej podsieci. Routery mogą obecnie wysyłać pakiety tylko do bezpośrednio połączonych urządzeń.

\subsection*{Router R1}
\inputminted[label=\#show cdp neighbors, firstline=156, lastline=157]{text}{R1.txt}%
\vspace{-1cm}%
\inputminted[label=\#show ip route, firstline=167, lastline=170]{text}{R1.txt}

\subsection*{Router R2}
\inputminted[label=\#show cdp neighbors, firstline=297, lastline=300]{text}{R2_1.txt}%
\vspace{-1cm}%
\inputminted[label=\#show ip route, firstline=310, lastline=316]{text}{R2_1.txt}

\subsection*{Router R3}
\inputminted[label=\#show cdp neighbors, firstline=215, lastline=218]{text}{R3.txt}%
\vspace{-1cm}%
\inputminted[label=\#show ip route, firstline=203, lastline=208]{text}{R3.txt}

\subsection*{Router R4}
\inputminted[label=\#show cdp neighbors, firstline=203, lastline=206]{text}{R4.txt}%
\vspace{-1cm}%
\inputminted[label=\#show ip route, firstline=173, lastline=178]{text}{R4.txt}

\subsection*{Router R5}
\inputminted[label=\#show cdp neighbors, firstline=221, lastline=223]{text}{R5.txt}%
\vspace{-1cm}%
\inputminted[label=\#show ip route, firstline=232, lastline=237]{text}{R5.txt}

\section{Wstępna konfiguracja protokołu OSPF}

Aby móc wysyłać pakiety IP po sieci, potrzebny jest protokół trasowania taki, jak OSPF. Na razie cała sieć jest w obszarze 0. Należy także skonfigurować interfejsy loopback komendą \mintinline{text}{ip address <loopback ip> 255.255.255.255} -- ich adresy IP będą one użyte przez Cisco IOS jako identyfikatory routerów. Protokół OSPF został skonfigurowany na każdym z routerów komendą \mintinline{text}{network <prefix> <wildcard-mask> area 0}.

\inputminted[label=Router R1, firstline=228, lastline=279]{text}{R1.txt}
\inputminted[label=Router R2, firstline=366, lastline=422]{text}{R2_1.txt}
\inputminted[label=Router R3, firstline=286, lastline=341]{text}{R3.txt}
\inputminted[label=Router R4, firstline=269, lastline=324]{text}{R4.txt}
\inputminted[label=Router R5, firstline=285, lastline=339]{text}{R5.txt}

\section{Baza danych OSPF}

\inputminted[label=Router R5, firstline=366, lastline=595]{text}{R5.txt}

\section{Wieloobszarowy OSPF}

\inputminted[label=Router R1, firstline=309, lastline=390]{text}{R1.txt}
\inputminted[label=Router R5, firstline=597, lastline=680]{text}{R5.txt}

\section{Koszty łącza OSPF}

\inputminted[label=Router R4 - \#ping 192.168.11.1, firstline=377, lastline=379]{text}{R4.txt}
\inputminted[label=Router R4 - \#traceroute 192.168.11.1, firstline=394, lastline=397]{text}{R4.txt}

\inputminted[label=Router R2 - koszty łączy, firstline=247, lastline=251]{text}{R2.txt}

\inputminted[label=Router R2 - koszty łączy po zmianie, firstline=325, lastline=372]{text}{R2.txt}

\inputminted[label=Router R4 - \#ping 192.168.11.1, firstline=535, lastline=537]{text}{R4.txt}
\inputminted[label=Router R4 - \#traceroute 192.168.11.1, firstline=541, lastline=545]{text}{R4.txt}

\section{Redystrybucja tras}

\inputminted[label=Router R1 - \#show ip route, firstline=838, lastline=844]{text}{R1.txt}
\inputminted[label=Router R5 - \#show ip route, firstline=718, lastline=728]{text}{R5.txt}

\inputminted[label=Router R5 - stan końcowy, firstline=729, lastline=817]{text}{R5.txt}

\end{document}
