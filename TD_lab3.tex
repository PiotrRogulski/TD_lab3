\documentclass[a4paper,12pt,notitlepage]{article}

\usepackage[svgnames]{xcolor}
\usepackage{amsmath, amssymb, amsthm}
\usepackage{mathtools}
\usepackage[math]{fontspec}
\usepackage[nosingleletter, lastparline]{impnattypo}
\usepackage[polish]{babel}
\usepackage{bookmark}
\usepackage[margin=1in]{geometry}
\usepackage[babel=true, tracking=true]{microtype}
\usepackage{minted}
\usepackage{array}
\usepackage{colortbl}
\usepackage{parskip}

\renewcommand*{\thesection}{\Alph{section}}

\definecolor{bg}{rgb}{0.95,0.95,0.95}
\setminted{frame=single, bgcolor=bg, breaklines=true}

\linespread{1.3}

\title{\textbf{TD -- laboratorium 3}}
\author{Piotr Rogulski 305867 \\ Szymon Sieradzki 305881}
\date{\today}

\IfFontExistsTF{JetBrainsMono-Regular}{
    \setmonofont{JetBrainsMono}[
        UprightFont = *-Light,
        BoldFont = *-Regular,
        ItalicFont = *-Light-Italic,
        Scale = MatchLowercase
    ]
}{}

\begin{document}

\maketitle
\section*{Schemat adresacji}

\begin{table}[!htb]
    \caption{Adresacja interfejsów}
    \centering
    \begin{tabular}{*4c}
        \hline\hline
            \textbf{Router} & \textbf{Interfejs} & \textbf{Adres IP} & \textbf{Podsieć} \\
        \hline
            \rowcolor{bg}
            R1 & e0/0 & 192.168.11.1  & 192.168.11.0/30 \\
        \hline
            \rowcolor{bg}
            R2 & e0/0 & 192.168.11.2  & 192.168.11.0/30 \\
               & e0/1 & 192.168.10.1  & 192.168.10.0/30 \\
               & e0/2 & 192.168.10.5  & 192.168.10.4/30 \\
        \hline
            R3 & e0/0 & 192.168.10.1  & 192.168.10.0/30 \\
               & e0/2 & 192.168.10.6  & 192.168.10.4/30 \\
               & e0/3 & 192.168.10.17 & 192.168.10.16/30 \\
        \hline
            R4 & e0/1 & 192.168.10.2  & 192.168.10.0/30 \\
               & e0/2 & 192.168.10.5  & 192.168.10.4/30 \\
               & e0/3 & 192.168.10.18 & 192.168.10.16/30 \\
        \hline
            R5 & e0/0 & 192.168.10.2  & 192.168.10.0/30 \\
               & e0/2 & 192.168.10.6  & 192.168.10.4/30 \\
        \hline\hline
    \end{tabular}
\end{table}

\begin{table}[!htb]
    \caption{Adresacja interfejsów loopback}
    \centering
    \begin{tabular}{*2c}
        \hline\hline
            \textbf{Router} & \textbf{Loopback IP} \\
        \hline
            R1 & 192.168.0.1 \\
            R2 & 192.168.0.2 \\
            R3 & 192.168.0.3 \\
            R4 & 192.168.0.4 \\
            R5 & 192.168.0.5 \\
        \hline\hline
    \end{tabular}
\end{table}

\section{Podstawowa konfiguracja urządzenia}

W początkowej konfiguracji urządzeń każdemu interfejsowi na każdym routerze przypisano adres IP komendą \mintinline{text}{ip address <ip address> <mask>} a następnie potwierdzono pomyślne przypisanie komendami \mintinline{text}{show cdp neighbors} oraz \mintinline{text}{show ip route}.

Na każdym z routerów wpisów w tablicy trasowania jest tyle, ile skonfigurowanych interfejsów, każdy w innej podsieci. Routery mogą obecnie wysyłać pakiety tylko do bezpośrednio połączonych urządzeń.

\subsection*{Router R1}
\inputminted[label=\#show cdp neighbors, firstline=156, lastline=157]{text}{R1.txt}%
\vspace{-1cm}%
\inputminted[label=\#show ip route, firstline=167, lastline=170]{text}{R1.txt}

\subsection*{Router R2}
\inputminted[label=\#show cdp neighbors, firstline=297, lastline=300]{text}{R2_1.txt}%
\vspace{-1cm}%
\inputminted[label=\#show ip route, firstline=310, lastline=316]{text}{R2_1.txt}

\subsection*{Router R3}
\inputminted[label=\#show cdp neighbors, firstline=215, lastline=218]{text}{R3.txt}%
\vspace{-1cm}%
\inputminted[label=\#show ip route, firstline=203, lastline=208]{text}{R3.txt}

\subsection*{Router R4}
\inputminted[label=\#show cdp neighbors, firstline=203, lastline=206]{text}{R4.txt}%
\vspace{-1cm}%
\inputminted[label=\#show ip route, firstline=173, lastline=178]{text}{R4.txt}

\subsection*{Router R5}
\inputminted[label=\#show cdp neighbors, firstline=221, lastline=223]{text}{R5.txt}%
\vspace{-1cm}%
\inputminted[label=\#show ip route, firstline=232, lastline=237]{text}{R5.txt}

\section{Wstępna konfiguracja protokołu OSPF}

Aby móc wysyłać pakiety IP po sieci, potrzebny jest protokół trasowania taki, jak OSPF. Na razie cała sieć jest w obszarze 0. Należy także skonfigurować interfejsy loopback komendą \mintinline{text}{ip address <loopback ip> 255.255.255.255} -- ich adresy IP będą one użyte przez Cisco IOS jako identyfikatory routerów. Protokół OSPF został skonfigurowany na każdym z routerów komendą \mintinline{text}{network <prefix> <wildcard-mask> area 0}.

\inputminted[label=Router R1, firstline=228, lastline=279]{text}{R1.txt}
\inputminted[label=Router R2, firstline=366, lastline=422]{text}{R2_1.txt}
\inputminted[label=Router R3, firstline=286, lastline=341]{text}{R3.txt}
\inputminted[label=Router R4, firstline=269, lastline=324]{text}{R4.txt}
\inputminted[label=Router R5, firstline=285, lastline=339]{text}{R5.txt}

\section{Baza danych OSPF}

Domyślnie wszystkie połączenia między routerami mają skonfigurowany typ broadcastowy, w rzeczywistości są one jednak point-to-point. By uniknąć zbędnego wyznaczania routerów DR i BDR, należy zmienić typ połączenia na odpowiednich interfejsach przy pomocy \mintinline{text}{ip ospf network point-to-point}. Point-to-point został skonfigurowany na połączeniach R2--R3, R2--R4, R4--R5 i R3--R5.

W bazie danych OSPF znajdują się $5$ wiadomości router LSA i $2$ wiadomości network LSA. Wiadomości router LSA R5 otrzymuje od każdego routera w obszarze, w tym od samego siebie, dlatego jest ich $5$, network LSA wysyłają routery DR, w tym przypadku R1--R2 i R3--R4 nie są połączone point-to-point, dlatego R1 i R3 są routerami DR.

Analizując informacje zwracane przez \mintinline{text}{show ip ospf database router} można zauważyć, że router R3 (adres loopback 192.168.0.3) jest połączony z samym sobą (adres 192.168.10.17) jako DR oraz z routerami R2 i R5 przez point-to-point.

\inputminted[label=Router R5, firstline=366, lastline=595]{text}{R5.txt}

\section{Wieloobszarowy OSPF}

W celu stworzenia wieloobszarowego OSPF zmieniony zostaje obszar interfejsu R1--R2 na 1 przy użyciu komendy \mintinline{text}{network <prefix> <wildcard-mask> area 1}. Router R2 jako jedyny posiada interfejsy w obydwóch obszarach, więc staje się routerem ABR. W celu weryfikacji topologii sieci na routerze R1 (obszar 1) i R5 (obszar 0) zostają wykonane \mintinline{text}{show ip ospf database} i \mintinline{text}{show ip ospf database summary}.

Na podstawie informacji z wywołania komendy \mintinline{text}{show ip ospf database} R1 otrzymuje router LSA z R1 i R2, a R5 otrzymuje router LSA z R2, R3, R4, R5. R1 jest dodatkowo połączony przez R2(ABR) z podsieciami obszaru 0, R5 jest połączony przez R2 z podsiecią R1--R2.

Komenda \mintinline{text}{show ip ospf database summary} pozwala uzyskać więcej informacji na temat summary network LSA -- wiadomości uzyskiwanych od routerów ABR, oznaczających podsieci z innego obszaru. Dla R1 są to podsieci 192.168.10.0, 192.168.10.4 i 192.168.10.16, dla R5 jest to 192.168.11.0.
\inputminted[label=Router R1, firstline=309, lastline=390]{text}{R1.txt}
\inputminted[label=Router R5, firstline=638, lastline=680]{text}{R5.txt}

\section{Koszty łącza OSPF}

W celu weryfikacji czy i jakimi ścieżkami routery przesyłają między sobą informacje, z routera R4 spingowany zostaje interfejs e0/0 routera R1 przy użyciu komend \mintinline{text}{ping <ip-address>} oraz \mintinline{text}{traceroute <ip-address>}. Pingowanie kończy się sukcesem, dane zostają przesłane.

Ścieżka przesyłu podana przez traceroute zgadza się z topologią sieci -- dane przesyłane są na router R2 (adres 192.168.10.1 dla interfejsu R2--R4), a następnie na R1 (adres 192.168.11.1).

\begin{table}[H]
    \caption{Koszty połączeń z routera R1}
    \label{tab:R1cost}
    \centering
    \begin{tabular}{*2c}
        \hline\hline
            \textbf{Podsieć} & \textbf{Koszt połączenia} \\
        \hline
            192.168.0.0/32 (loopback) & bezpośrednio połączony \\
            192.168.11.0/30 & bezpośrednio połączony \\
            192.168.10.0/30 & 20 \\
            192.168.10.4/30 & 20 \\
            192.168.10.16/30 & 30 \\
        \hline\hline
    \end{tabular}
\end{table}

Koszty połączeń zestawione z tablicy (\ref{tab:R1cost}) wynikają z narzuconej topologii sieci: interfejs e0/0 routera R1 jest skonfigurowany w podsieci 192.168.11.0/30, podsieci 192.168.10.0/30 oraz 192.168.10.4/30 (odpowiednio dla połączeń R2--R4 oraz R2--R3) są dostępne przez dwa przeskoki (każde o metryce 10) a podsieć 192.168.10.16 jest dostępna przez trzy.

Dopóki wszystkie połączenia miały przypisany koszt równy 10, połączenie R2--R4 miało koszt 10 (routery te są bezpośrednio połączone interfejsami e0/1). Po wzroście kosztu tego połączenia routery R2 i R4 zaczęły komunikować się ścieżką o koszcie 20 -- poprzez router R3.

Koszt połączenia R2--R4 zostaje ustawiony na obu routerach na 100, zmiana ta jest widoczna w \mintinline{text}{show ip ospf interface} wywołanym na R2. Ponowne spingowanie R1 z R4 przy pomocy traceroute daje inny rezultat -- OSPF wybiera najkrótszą ścieżkę, a ta nie wiedzie teraz przez R2--R4. Zgodnie z topologią sieci najkrótsza ścieżka to teraz R4--R3--R2--R1.

\inputminted[label=Router R4 - \#ping 192.168.11.1, firstline=377, lastline=379]{text}{R4.txt}
\inputminted[label=Router R4 - \#traceroute 192.168.11.1, firstline=394, lastline=397]{text}{R4.txt}

\inputminted[label=Router R2 - koszty łączy, firstline=247, lastline=251]{text}{R2.txt}

\inputminted[label=Router R2 - koszty łączy po zmianie, firstline=325, lastline=372]{text}{R2.txt}

\inputminted[label=Router R4 - \#ping 192.168.11.1, firstline=535, lastline=537]{text}{R4.txt}
\inputminted[label=Router R4 - \#traceroute 192.168.11.1, firstline=541, lastline=545]{text}{R4.txt}

\section{Redystrybucja tras}
Router R1 jest połączony bezpośrednio wyłącznie z R2, uzasadnionym jest zatem zmiana protokołu połączenia R1-R2 z OSPF na RIP. W tym celu OSPF zostaje wyłączony na R1, a RIP zostaje włączony na R1 i R2, na nowo skonfigurowane zostają adresy. Komenda \mintinline{text}{show ip route} wywołana na R5 wskazuje, że interfejs R1-R2 przestał być interfejsem inter-area. Zgadza się to z wprowadzonymi zmianami, ponieważ R1-R2 nie tworzą teraz obszaru OSPF. 

Komenda \mintinline{text}{show ip ospf database router} pokazuje z kolei, że R2 z ID 192.168.0.2 jest ASBR routerem, na podstawie  \mintinline{text}{show ip ospf database external} wiadomo, że łączy on R5 z siecią 192.168.11.0 (R2 jest Advertising Routerem), co zgadza się z założoną topologią sieci. Metryka dostępu do tej sieci wynosi 100, wynika to z faktu, że na R2 ustawiliśmy default-metric na 100, przy usunięciu R1-R2 z OSPF. Wówczas zaszła zmiana topologii sieci więc trasa do R5 została zredystrybuowana. Przy drugiej zmianie metryki, nie doszło do zmiany w obszarze OSPF, więc metryka dla R5 się nie zmieniła.
\inputminted[label=Router R1 - \#show ip route, firstline=838, lastline=844]{text}{R1.txt}
\inputminted[label=Router R5 - \#show ip route, firstline=718, lastline=728]{text}{R5.txt}

\inputminted[label=Router R5 - stan końcowy, firstline=729, lastline=817]{text}{R5.txt}

\end{document}
